\documentclass{article}

\usepackage[utf8]{inputenc}
\usepackage[T1]{fontenc}
\usepackage{lmodern}
\usepackage{tikz}
\usepackage{listings}
\usepackage{color}
\usepackage{enumitem}
\usepackage{mathtools}
\usepackage{amsmath}
\usepackage{amsfonts}

\title{BuK Abgabe 4 | Gruppe 17}
\author{Malte Meng (354529) , Charel Ernster (318949), Sebastian Witt (354738)}
\begin{document}
	\maketitle 
	\section[a 4.1]{Aufgabe 4.1}
		Wir nehmen an $L_{self}$ sei rekursiv.\\
		Somit gibt es die TM bzw Gödelnummer $\langle M_{self}\rangle$ die $L_{self}$ entscheidet.\\
		Wird nun diese Turing Maschine auf der eigenen Eingabe ausgeführt kommt es zu folgendem Widerspruch:\\\\
		Fall 1:\\
		$\langle M_{self}\rangle \in L_{self} \xRightarrow{Def. von L_{self}} M_{self}$ verwirft $\langle M_{self}\rangle \\
		\xRightarrow{M_{self} entscheidet L_{self}} \langle M_{self}\rangle \notin L_{self}$ (widerspruch)\\
		Fall 2:\\
		$\langle M_{self}\rangle \notin L_{self} \xRightarrow{Def. von L_{self}} M_{self}$ akzeptiert $\langle M_{self}\rangle\\ \xRightarrow{M_{self} entscheidet L_{self}} \langle M_{self}\rangle \in L_{self}$\\\\
		Somit ist $L_{self}$ nicht rekursiv.\\
	\section[a 4.2]{Aufgabe 4.2}
		\begin{enumerate} [label=\alph*.]
			\item Sei S = $\{f_M|f_M(x) = \bot, x \in \sum^*\}$\\
			S ist nicht Trivial, da eine Turingmaschine A mit folgendem Übergang Konstruiert werden kann: \\$(q_r,0) \rightarrow (q_r,0,N)$.\\ Somit A $\in$ S $\implies S \neq \{\}$\\
			S $\neq$ R da für $\{f_M|f_M(x) = x\}$ gilt $f_M \in R$, $f_M \notin S$.\\
			Somit gilt der Satz von Rice:\\
			$L(S) = \{\langle M\rangle$ | M berechnet eine Funktion aus S$\}$\\
			$= \{\langle M\rangle $| M stoppt auf keiner Eingabe$\}$\\
			Gemäß des Satz von Rice ist $\textbf{H}_{never}$ nicht entscheidbar.\\\\
			\item Sei S = $\{f_M|f_M(101)$ besucht den Zustand $q_{15}\}$\\
			S ist nicht trivial, da jede TM durch umbenennung des Startzustandes "$q_{start} \rightarrow q_{15}$" den Zustand $q_{15}$ besucht. S ist aber auch nicht R, da bei jeder TM der Zustand $q_{15}$ umbenannt werden kann.\\
			Somit gilt der Satz von Rice:\\
			$L(S) = \{\langle M\rangle$ | M berechnet eine Funktion aus S$\}$\\
			$= \{\langle M\rangle $| M besucht bei Eingabe 101 den Zustand $ q_{15}\}$\\
			Gemäß des Satz von Rice ist $\textbf{S}_{15}$ nicht entscheidbar.\\
		\end{enumerate}
\end{document}