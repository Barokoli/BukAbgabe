\documentclass{article}

\usepackage[utf8]{inputenc}
\usepackage[T1]{fontenc}
\usepackage{lmodern}
\usepackage{tikz}
\usepackage{listings}
\usepackage{color}
\usepackage{enumitem}
\usepackage{mathtools}
\usepackage{amsmath}
\usepackage{amsfonts}

\title{BuK Abgabe 4 | Gruppe 17}
\author{Malte Meng (354529) , Charel Ernster (318949), Sebastian Witt (354738)}
\begin{document}
	\maketitle 
	\section[a 4.1]{Aufgabe 4.1}
		Wir nehmen an $L_{self}$ sei rekursiv.\\
		Somit gibt es die TM bzw Gödelnummer $\langle M_{self}\rangle$ die $L_{self}$ entscheidet.\\
		Wird nun diese Turing Maschine auf der eigenen Eingabe ausgeführt kommt es zu folgendem Widerspruch:\\\\
		Fall 1:\\
		$\langle M_{self}\rangle \in L_{self} \xRightarrow{Def. von L_{self}} M_{self}$ verwirft $\langle M_{self}\rangle \\
		\xRightarrow{M_{self} entscheidet L_{self}} \langle M_{self}\rangle \notin L_{self}$ (widerspruch)\\
		Fall 2:\\
		$\langle M_{self}\rangle \notin L_{self} \xRightarrow{Def. von L_{self}} M_{self}$ akzeptiert $\langle M_{self}\rangle\\ \xRightarrow{M_{self} entscheidet L_{self}} \langle M_{self}\rangle \in L_{self}$\\\\
		Somit ist $L_{self}$ nicht rekursiv.\\
\end{document}