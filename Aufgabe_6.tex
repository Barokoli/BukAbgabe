\documentclass{article}

\usepackage[utf8]{inputenc}
\usepackage[T1]{fontenc}
\usepackage{lmodern}
\usepackage{tikz}
\usepackage{listings}
\usepackage{color}
\usepackage{enumitem}
\usepackage{mathtools}
\usepackage{amsmath}
\usepackage{amsfonts}
\newlist{arrowlist}{itemize}{1}
\setlist[arrowlist]{label=$\Rightarrow$}

\title{BuK Abgabe 6 | Gruppe 17}
\author{Malte Meng (354529) , Charel Ernster (318949), Sebastian Witt (354738)}
\begin{document}
	\maketitle 
	\section[a 6.1]{Aufgabe 6.1}
	\section[a 6.2]{Aufgabe 6.2}
	Die PKP-Instanz ist lösbar:\\\\
	$
	\Big[ \frac{bb}{bba} \Big]
	\Big[ \frac{aa}{ab} \Big]
	\Big[ \frac{bb}{bba} \Big]
	\Big[ \frac{baab}{ab} \Big]
	=
	\Big[ \frac{bbaabbbaab}{bbaabbbaab}\Big]
	$\\
	\section[a 6.3]{Aufgabe 6.3}
	\begin{enumerate}[label=(\alph*).]
		\item Dieses spezielle PKP-Problem ist lösbar, da sich eine Turingmaschine konstruieren lässt die per "brute force" alle konfigurationen der Länge $\leq l$  ausprobiert.
	\end{enumerate}
\end{document}