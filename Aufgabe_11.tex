\documentclass{article}

\usepackage[utf8]{inputenc}
\usepackage[T1]{fontenc}
\usepackage{lmodern}
\usepackage{tikz}
\usepackage{listings}
\usepackage{color}
\usepackage{enumitem}
\usepackage{mathtools}
\usepackage{amsmath}
\usepackage{amsfonts}
\newlist{arrowlist}{itemize}{1}
\setlist[arrowlist]{label=$\Rightarrow$}

\title{BuK Abgabe 11 | Gruppe 17}
\author{Malte Meng (354529) , Charel Ernster (318949), Sebastian Witt (354738)}
\begin{document}
	\maketitle 
	\section[a 11.1]{Aufgabe 11.1}
	Zeige dass 3SAT $\leq_p$ NAESAT (Not-All-Equal-SAT):\\\\
	\textbf{Reduktionsabbildung:}\\
	Sei $\phi$ eine Formel in 3KNF.
	$h$ ist die Funktion, die eine Formel in 3KNF, in folgende Form in 4KNF bringt:\\
	Für $k_i = (x_{0} \vee x_{1} \vee x_{2})$ ist Klausel von $\phi$.\\
	$h(\phi)$ = \{$\forall k_i \in \phi$ | $k_i$ = $(x_0 \vee x_1 \vee x_2 \vee \neg x_2')$\}.\\
	$g$ ist die Funktion, die eine Formel in 4KNF ($\beta$), in folgende Form in 3KNF bringt:\\
	Für $k_i' = (x_0 \vee x_1 \vee x_2 \vee \neg x_2')$ ist Klausel von $\beta$.\\
	$g(\beta)$ = \{$\forall k_i \in \beta$ | $k_i$ = 
	$(x_0 \vee x_1 \vee x_i') \wedge (\neg x_i' \vee x_2 \vee \neg x_2')$\}.\\
	$f$ sei die hintereinanderausführung von $h$ und $g$.\\ 
	$f = g \circ h$\\
	$h$ ist in linearer Zeit berechenbar (abhängig von der Anzahl von Klauseln).\\
	$g$ ist in linearer Zeit berechenbar (abhängig von der Anzahl von Klauseln).\\
	$h$ linear berechenbar $\wedge$ $g$ linear berechenbar $\implies g\circ h$ linear berechenbar $\Leftrightarrow f$ ist linear berechenbar.\\\\
	\textbf{Korrektheit:}\\
	Zu Zeigen:\\
	$f(\phi) \in$ NAESAT $\implies$ $\phi \in$ 3SAT\\
	$f(\phi) \notin$ NAESAT $\implies$ $\phi \notin$ 3SAT\\\\
	$f(\phi) \in$ NAESAT $\implies$ Es gibt eine erfüllende Belegung $\sigma$ für alle Klauseln $k$ in $f(\phi)$ und jede Klausel enthält ein wahres und ein falsches Literal. 
	
	
	
\end{document}
