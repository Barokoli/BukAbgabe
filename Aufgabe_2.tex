\documentclass{article}

\usepackage[utf8]{inputenc}
\usepackage[T1]{fontenc}
\usepackage{lmodern}
\usepackage{tikz}
\usepackage{listings}
\usepackage{color}
\usepackage{enumitem}
\usepackage{mathtools}
\usepackage{amsmath}
\usepackage{amsfonts}

\title{BuK Abgabe 2 | Gruppe 17}
\author{Malte Meng (354529) , Charel Ernster (318949), Sebastian Witt (354738)}
\begin{document}
	\maketitle 
	\section[a 2.1]{Aufgabe 2.1}
	Gödelnummer $\langle M\rangle$:\\
	$\langle M\rangle = 111 0101000100100 11 010010101000 11 01000100100010 11 00010101010 11 000100100010010 111 $
	\section[a 2.2]{Aufgabe 2.2}
	Der Speicherbedarf ist auf die Länge der Eingabe begrenzt.\\
	Annahme I: "Es darf nicht zu Wiederholungen kommen, da die Maschine sonst nicht hält"\\
	I $\implies$ TM kann nur einmal in jedem Zustand sein.\\\\
	II: Zustände des Bandes: $|\Gamma|^{s(n)}$ \\
	III: Zustände der TM (-1 für Endzustand): $(|Q| - 1)$\\
	IV: Lesekopfpositionen: $s(n)$\\\\
	Das Produkt dieser Zustände sind die maximal Möglichen Konfigurationen einer haltenden TM. Zuzüglich des Haltens am Ende.(+1)\\\\
	$|II \cap III \cap IV| +1 = (|Q| - 1) * |\Gamma|^{s(n)} * s(n)+1$
	\quad$\square$\\
	
\end{document}