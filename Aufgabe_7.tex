\documentclass{article}

\usepackage[utf8]{inputenc}
\usepackage[T1]{fontenc}
\usepackage{lmodern}
\usepackage{tikz}
\usepackage{listings}
\usepackage{color}
\usepackage{enumitem}
\usepackage{mathtools}
\usepackage{amsmath}
\usepackage{amsfonts}
\newlist{arrowlist}{itemize}{1}
\setlist[arrowlist]{label=$\Rightarrow$}

\title{BuK Abgabe 7 | Gruppe 17}
\author{Malte Meng (354529) , Charel Ernster (318949), Sebastian Witt (354738)}
\begin{document}
	\maketitle 
	\section[a 7.1]{Aufgabe 7.1}
	Die RAM-Befehler können folgendermaßen ersetzt werden:\\
	\begin{enumerate}[label=(\alph*).]
		\item MULT 1:\\
		\lstset{
			numbers=left,
			stepnumber=1,    
			firstnumber=0,
			numberfirstline=false
		}
		\begin{lstlisting}
		//Multipliziere c(0) und c(1) Ergebniss in c(0)
		STORE j
		STORE j+2
		LOAD 1
		STORE j+1
		STORE j+3
		if c(0) = 0 THEN GOTO 23
			LOAD j+2
			if c(0) = 0 THEN GOTO 16
				LOAD j+4
				CADD 1
				STORE j+4
				CLOAD j+2
				CADD -1
				STORE j+2
				GOTO 8
			END
			LOAD j
			STORE j+2
			LOAD j+3
			CADD -1
			STORE j+3
			GOTO 6
		END
		LOAD j+4
		\end{lstlisting}
		\item INDLOAD i:
		\begin{lstlisting}
		//Load c(c(i))
		LOAD i
		LOAD c(0)
		\end{lstlisting}
	\end{enumerate}
	
	\section{7.2}

\begin{itemize}
\item (a)\\
Dieses Problem ist entscheidbar. Nach der Definition von LOOP-Programmen sind jene immer endlich, also terminieren nach einer endlichen Anzahl an Schritten. Man kann also die LOOP-Schleife simulieren und erhält nach endlich vielen Schritten die Rückgabe des LOOP-Programms und kann somit entscheiden, ob das Programm auf die Eingabe $x$ die Ausgabe $y$ berechnet.

\item (b)\\
Angenommen das Problem wäre entscheidbar. Da WHILE-Programme turinvollständig sind, muss auch $L$ entscheidbar sein:\\

$L=\{\langle M \rangle \mid M$ berechnet zur Eingabe $x$ die Ausgabe $y \}$\\

$L$ lässt sich durch ein WHILE-Programm berechnen. $L$ ist  jedoch unentscheidbar, dies zeigen wir mittels Satz von Rice.

Sei $S:=\{f_M \mid f_M(x) = y \}$\\

Offensichtlich ist $S=\emptyset$. Ebenfalls gilt $S \neq R$, denn es gilt $f_{\bot} \in R \setminus S$ mit $f_{\bot}(w)=\bot$ für alle $w \in \Sigma$. Daraus folgt:\\

\begin{center}
$L(S) = \{ \langle M \rangle \mid M$ berechnet eine Funktion aus $S \}$\\
 $= \{ \langle M \rangle \mid M$ berechnet zur Eingabe $x$ die Ausgabe $y \} = L$\\
\end{center}

Aufgrund des Satzes von Rice ist $L$ unentscheidbar.
\end{itemize}
	
	\section[a 7.3]{Aufgabe 7.3}
	Aussage (a) trifft zu, da $A_{LOOP}$ auf $H$ nach folgendem schema Reduzierbar ist. \\
	Berechenbare Funktion f:\\
	$\langle P \rangle$ ist ungültiges LOOP-Programm $\rightarrow$ Touringmaschine die nie hält.\\
	Ansonsten $\rightarrow$ Touringmaschine die das LOOP-Programm simuliert (hält immer).\\
	Dies ist aber auch schon dadurch gegeben, dass die LOOP-Programme rekursiv berechenbar sind und das Halteproblem nur rekursiv aufzählbar ist, dies schließt (b) aus.
	\section[a 7.4]{Aufgabe 7.4}
	
\end{document}
