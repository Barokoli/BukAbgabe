\documentclass{article}

\usepackage[utf8]{inputenc}
\usepackage[T1]{fontenc}
\usepackage{lmodern}
\usepackage{tikz}
\usepackage{listings}
\usepackage{color}
\usepackage{enumitem}
\usepackage{mathtools}
\usepackage{amsmath}
\usepackage{amsfonts}

\title{BuK Abgabe 1 | Gruppe 17}
\author{Malte Meng (354529) , Sebastian Witt (354738)}
\begin{document}
	\maketitle 
	\section[a 1.1]{Aufgabe 1.1}
	\begin{enumerate} [label=\alph*.]
		\item \textbf{Das Teilsummenproblem:}\\
		L = \{(M,b) | M $\subset \mathbb{N} , b \in \mathbb{N} , \exists P \in Pot(M), n \in \mathbb{N}, n = |P|, \big(\sum_{i=1}^{n} P_i \big)= b$\}
		\item \textbf{Cliquenproblem:}\\
		$L_{Clique} = \{ (G,b)\ |\ b \in \mathbb{N},G = (V,E),\exists v \in V, n(v) = b, n \in V \rightarrow \mathbb{N}, n \equiv Anzahl\ der\ Nachbarn\ von\ v \}$
	\end{enumerate}
	\section[a 1.2]{Aufgabe 1.2}
	Geben Sie zu der folgenden Turingmaschine M an, welche Konfigurationen auf der Eingabe \textit{w} = 110 erreicht werden.
	\\\\
	Konfigurationen K:\\
	K =  \{$q_0110 , 1q_010 , 11q_00 , 110q_0 , 11q_10 , 110\bar{q}$\}
	\section[a 1.3]{Aufgabe 1.3}
	Die Turingmaschine M verhält sich wie folgt:\\
	\begin{enumerate}[label=-]
		\item Wenn die Maschine auf leerer Eingabe läuft wird Abgelehnt.\\
		\item Die TM Akzeptiert Eingaben mit unterschiedlichen Zeichen am Anfang und Ende.
	\end{enumerate}
	\pagebreak
	\section[a 1.3]{Aufgabe 1.3}
	Turingmaschine M:\\
	M = $(\{q_0,q_1,q_2,q_3,q_4\},\{0,1\},\{0,1,B\},B,q_0,\bar{q},\delta)$\\\\
	\begin{tabular}{l || c c c}
		$\delta$ & 0 & 1 & B\\
		\hline
		$q_0$ & $(q_1,0,R)$ & $(q_1,1,R)$ & $(\bar{q},B,N)$\\
		$q_1$ & $(q_1,0,R)$ & $(q_1,1,R)$ & $(q_2,B,L)$\\
		$q_2$ & $(q_3,0,L)$ & $(q_3,1,L)$ & $(\bar{q},B,N)$\\
		$q_3$ & $(q_4,1,L)$ & $(q_3,1,L)$ & $(q_4,1,N)$\\
		$q_4$ & $(q_4,0,L)$ & $(q_4,1,L)$ & $(\bar{q},B,R)$\\
	\end{tabular}\\\\
	Die TM M funktioniert nach folgendem Schema:\\
	Ist die Eingabe leer ($\epsilon$) wird sofort der Endzustand erreicht und $\epsilon$ zurückgegeben.\\
	Ist sie nicht leer, wird zunächst ans Ende der Eingabe manövriert. ($q_1\ \rightarrow\ q_2$)\\
	Dann springt der Kopf an vorletzte Stelle. ($q_2\ \rightarrow\ q_3$)\\
	Nun wird bei einer Null, Eins geschrieben und in den $q_4$ Zustand gewechselt. Bei einer Eins wird der Übertrags Zustand eingenommen $q_3$\\
	Wenn der Übertrag geschrieben wurde (vllt. auch vor die ursprüngliche Eingabe) wird in $q_4$ an die linksmögliche Stelle bis zu einem Blank manövriert und mit $\bar{q}$ Terminiert. W + 2 steht nun rechts neben dem Lesekopf.
\end{document}