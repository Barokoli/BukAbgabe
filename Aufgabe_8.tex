\documentclass{article}

\usepackage[utf8]{inputenc}
\usepackage[T1]{fontenc}
\usepackage[table]{xcolor}
\usepackage{lmodern}
\usepackage{tikz}
\usepackage{listings}
\usepackage{color}
\usepackage{enumitem}
\usepackage{mathtools}
\usepackage{amsmath}
\usepackage{amsfonts}
\newlist{arrowlist}{itemize}{1}
\setlist[arrowlist]{label=$\Rightarrow$}

\newcommand{\rt}{\color{red}}
\newcommand{\bt}{\color{blue}}
\newcommand{\rc}{\cellcolor{red!20}}
\newcommand{\bc}{\cellcolor{blue!20}}
\title{BuK Abgabe 8 | Gruppe 17}
\author{Malte Meng (354529) , Charel Ernster (318949), Sebastian Witt (354738)}
\begin{document}
	\maketitle 
	\section[a 8.2]{Aufgabe 8.2}
	\begin{enumerate}[label=(\alph*).]
		\item Graphen in Adjazenzschreibweise mit Colorierung:
		
		\begin{equation}
			\begin{array}{l | c | c | c | c | c | c | c}
				\ & \rt1 &  \rt2 & \bt3 & \bt4 & \rt5 & \bt6 & \rt7\\
				\hline
				\rt1& \rc0 & \rc0 & 1 & 0 & \rc0 & 0 & \rc0\\
				\rt2& \rc0 & \rc0 & 1 & 1 & \rc0 & 0 & \rc0\\
				\bt3& 1 & 1 & \bc0 & \bc0 & 1 & \bc0 & 0\\
				\bt4& 0 & 1 & \bc0 & \bc0 & 1 & \bc0 & 1\\
				\rt5& \rc0 & \rc0 & 1 & 1 & 0 & 1 & 0\\
				\bt6& 0 & 0 & \bc0 & \bc0 & 1 & 0 & 1\\
				\rt7& \rc0 & \rc0 & 0 & 1 & 0 & 1 & 0\\
			\end{array}
		\end{equation}
		Wie in der Matrix dargestellt können die Punkte erfolgreich 2-Coloriert werden.
		\begin{equation}
			\begin{array}{l | c | c | c | c | c | c | c}
				 & \rt1 &  \rt2 & \bt3 & \bt4 & \rt5 & \bt6 & \rt7\\
				\hline
				\rt1& \rc0 & \rc0 & 1 & 0 & \rc0 & 0 & \rc\textbf{\textcircled{1}}\\
				\rt2& \rc0 & \rc0 & 1 & 1 & \rc0 & 0 & \rc0\\
				\bt3& 1 & 1 & \bc0 & \bc0 & 1 & \bc0 & 0\\
				\bt4& 0 & 1 & \bc0 & \bc0 & 1 & \bc0 & 1\\
				\rt5& \rc0 & \rc0 & 1 & 1 & 0 & 1 & 0\\
				\bt6& 0 & 0 & \bc0 & \bc0 & 1 & 0 & 1\\
				\rt7& \rc\textbf{\textcircled{1}} & \rc0 & 0 & 1 & 0 & 1 & 0\\
			\end{array}
		\end{equation}
		Es kann keine korrekte 2-Färbung gefunden werden, da der Kreis um \{1,3,5,6,7\} eine ungerade Anzahl von Elementen hat. Somit lässt sich der gesamte graph nicht 2-Colorieren.
		
		\item Greedy Algorithmus für \textbf{2-Colorability}:\\
		\begin{enumerate}[label=\arabic*.]
			\item Coloriere einen Knoten.
			\item Sind Anliegende Knoten gleichfarbig wird der Graph abgelehnt.
			\item Anliegende Knoten ohne Farbe werden in komplementärer Farbe gefärbt und beginne bei Schritt 2. für diese Knoten.
			\item wenn alle Knoten erfolgreich gefärbt sind ist der Graph \textbf{2-Colorable}.
			\item wenn nicht alle Knoten coloriert sind, beginne bei 1. mit einem uncolorierten Knoten.
		\end{enumerate}
		Dieser Algorithmus ist Korrekt, da folgende Eigenschaften erfüllt sind:
		\begin{enumerate}[label=\Roman*.]
			\item Alle Knoten werden Coloriert. (siehe 4.)
			\item Es können keine zwei gleichfarbigen Knoten verbunden werden. (siehe 2)
			\item Es wird jede korrekte Konfiguration abgedeckt, da bei Schritt 1. die Farbe irrelevant ist und jederzeit alle Farben invertiert werden können. Durch Schritt 1. ist die Farbe von Graphen-komponenten unabhängig. Jede Komponente hat zwei mögliche Colorierungen (Invertierung). 
		\end{enumerate}
	\end{enumerate}

	\section[a 8.3]{Aufgabe 8.3}
	Um zu zeigen, dass 3-Colorability in NP liegt wird 3-Colorability $\leq_p$ SAT gezeigt.\\
	Es gibt eine Funktion f mit folgenden Eigenschaften welche in polynomieller Zeit berechenbar ist.\\
	Jede Farbkonfiguration aus dem Problem der 3-Colorability wird folgendermaßen im SAT-Problem konfiguriert: 
	\begin{align*}
	\forall v \in V, x^1_v,x^2_v,x^3_v,\text{mit} \begin{cases}
	i = \text{farbe}(1...3)|  x^i_v = 1\\
	i \neq \text{farbe}(1...3)|  x^i_v = 0
	\end{cases}\\
	\text{Knotenbedingung: }& \bigwedge_{v \in V} (x^1_v \vee x^2_v \vee x^3_v)\\
	\text{Kantenbedingung: }& \bigwedge_{\{u,v\} \in E} (\bar{x}^1_u \vee \bar{x}^1_v) \wedge ( \bar{x}^2_v \vee \bar{x}^2_u) \wedge (\bar{x}^3_u \vee \bar{x}^3_v)\\
	\varphi = \text{Knotenbedingung $\wedge$ Kantenbedingung}\\
	f konstruiert \varphi\\
	Anzahl der Literale: O(3*|V|+3*|E|)\\
	max(|E|) = |V^2|\\
	Anzahl der Literale: O(|V^3|)\\
	\end{align*}
	Somit ist die Länge der zu berechnenden Formel für f polynomiell beschränkt und kann also in polynomieller Zeit berechnet werden.\\
	
	Korrektheit:\\
	\begin{align*}
		\text{Graph G ist colorierbar.} \Rightarrow& \text{Es gibt 3-Färbung.}\\
		\Rightarrow& \text{Knotenbedingung ist erfüllt, da jeder Knoten eine Farbe hat.}\\
		\Rightarrow& \text{keine zwei benachbarten Kanten haben die selbe Farbe.}\\
		\Rightarrow& \text{Kantenbedingung ist erfüllt.}\\		
		\Rightarrow& \varPhi \text{ ist erfüllbar}\\
	\end{align*}
	\begin{align*}
	\varphi \text{ ist erfüllbar.} \Rightarrow& \text{Es gibt eine erfüllende Belegung für }\varphi.\\
	\Rightarrow& \text{Die Knotenbedingung und die Kantenbedingung sind erfüllt}\\
	\Rightarrow& \text{jeder Knoten hat mindestens eine Farbe}\\
	\Rightarrow& \text{keine zwei adjazenten Knoten sind gleichfarbig}\\		
	\Rightarrow& \text{es gibt eine korrekte 3-Färbung für G}\\
	\end{align*}
	Somit lässt sich 3-Colorability auf SAT Reduzieren. Also ist auch 3-Colorability in NP.
	$\square$
\end{document}