\documentclass{article}

\usepackage[utf8]{inputenc}
\usepackage[T1]{fontenc}
\usepackage[table]{xcolor}
\usepackage{lmodern}
\usepackage{tikz}
\usepackage{listings}
\usepackage{color}
\usepackage{enumitem}
\usepackage{mathtools}
\usepackage{amsmath}
\usepackage{amsfonts}
\newlist{arrowlist}{itemize}{1}
\setlist[arrowlist]{label=$\Rightarrow$}

\newcommand{\rt}{\color{red}}
\newcommand{\bt}{\color{blue}}
\newcommand{\rc}{\cellcolor{red!20}}
\newcommand{\bc}{\cellcolor{blue!20}}
\title{BuK Abgabe 8 | Gruppe 17}
\author{Malte Meng (354529) , Charel Ernster (318949), Sebastian Witt (354738)}
\begin{document}
	\maketitle 
	\section[a 8.1]{Aufgabe 8.1}
	\begin{enumerate}[label=(\alph*).]
		\item Graphen in Adjazenzschreibweise mit Colorierung:
		
		\begin{equation}
			\begin{array}{l | c | c | c | c | c | c | c}
				\ & \rt1 &  \rt2 & \bt3 & \bt4 & \rt5 & \bt6 & \rt7\\
				\hline
				\rt1& \rc0 & \rc0 & 1 & 0 & \rc0 & 0 & \rc0\\
				\rt2& \rc0 & \rc0 & 1 & 1 & \rc0 & 0 & \rc0\\
				\bt3& 1 & 1 & \bc0 & \bc0 & 1 & \bc0 & 0\\
				\bt4& 0 & 1 & \bc0 & \bc0 & 1 & \bc0 & 1\\
				\rt5& \rc0 & \rc0 & 1 & 1 & 0 & 1 & 0\\
				\bt6& 0 & 0 & \bc0 & \bc0 & 1 & 0 & 1\\
				\rt7& \rc0 & \rc0 & 0 & 1 & 0 & 1 & 0\\
			\end{array}
		\end{equation}
		Wie in der Matrix dargestellt können die Punkte erfolgreich 2-Coloriert werden.
		\begin{equation}
			\begin{array}{l | c | c | c | c | c | c | c}
				 & \rt1 &  \rt2 & \bt3 & \bt4 & \rt5 & \bt6 & \rt7\\
				\hline
				\rt1& \rc0 & \rc0 & 1 & 0 & \rc0 & 0 & \rc\textbf{\textcircled{1}}\\
				\rt2& \rc0 & \rc0 & 1 & 1 & \rc0 & 0 & \rc0\\
				\bt3& 1 & 1 & \bc0 & \bc0 & 1 & \bc0 & 0\\
				\bt4& 0 & 1 & \bc0 & \bc0 & 1 & \bc0 & 1\\
				\rt5& \rc0 & \rc0 & 1 & 1 & 0 & 1 & 0\\
				\bt6& 0 & 0 & \bc0 & \bc0 & 1 & 0 & 1\\
				\rt7& \rc\textbf{\textcircled{1}} & \rc0 & 0 & 1 & 0 & 1 & 0\\
			\end{array}
		\end{equation}
		Es kann keine korrekte 2-Färbung gefunden werden, da der Kreis um \{1,3,5,6,7\} eine ungerade Anzahl von Elementen hat. Somit lässt sich der gesamte graph nicht 2-Colorieren.
		
		\item Greedy Algorithmus für \textbf{2-Colorability}:\\
		\begin{enumerate}[label=\arabic*.]
			\item Coloriere einen Knoten.
			\item Sind Anliegende Knoten gleichfarbig wird der Graph abgelehnt.
			\item Anliegende Knoten ohne Farbe werden in komplementärer Farbe gefärbt und beginne bei Schritt 2. für diese Knoten.
			\item wenn alle Knoten erfolgreich gefärbt sind ist der Graph \textbf{2-Colorable}.
			\item wenn nicht alle Knoten coloriert sind, beginne bei 1. mit einem uncolorierten Knoten.
		\end{enumerate}
		Dieser Algorithmus ist Korrekt, da folgende Eigenschaften erfüllt sind:
		\begin{enumerate}[label=\Roman*.]
			\item Alle Knoten werden Coloriert. (siehe 4.)
			\item Es können keine zwei gleichfarbigen Knoten verbunden werden. (siehe 2)
			\item Es wird jede korrekte Konfiguration abgedeckt, da bei Schritt 1. die Farbe irrelevant ist und jederzeit alle Farben invertiert werden können. Durch Schritt 1. ist die Farbe von Graphen-komponenten unabhängig. Jede Komponente hat zwei mögliche Colorierungen (Invertierung). 
		\end{enumerate}
	\end{enumerate}
\end{document}