\documentclass{article}

\usepackage[utf8]{inputenc}
\usepackage[T1]{fontenc}
\usepackage[table]{xcolor}
\usepackage{lmodern}
\usepackage{tikz}
\usepackage{listings}
\usepackage{color}
\usepackage{enumitem}
\usepackage{mathtools}
\usepackage{amsmath}
\usepackage{amsfonts}
\newlist{arrowlist}{itemize}{1}
\setlist[arrowlist]{label=$\Rightarrow$}

\newcommand{\rt}{\color{red}}
\newcommand{\bt}{\color{blue}}
\newcommand{\rc}{\cellcolor{red!20}}
\newcommand{\bc}{\cellcolor{blue!20}}
\title{BuK Abgabe 8 | Gruppe 17}
\author{Malte Meng (354529) , Charel Ernster (318949), Sebastian Witt (354738)}
\begin{document}
	\maketitle 
	\section[a 8.1]{Aufgabe 8.1}
	\begin{enumerate}
		\item (a) Die TM wird in Phasen geteilt. Man gewinnt dadurch Zeit, dass man 2 Übergänge einfügt pro Speicherzelle und es die Option gibt ohne einen Vergleich von Symbolen von $v$ und $w$ zum nächsten Symbole nach rechts zu laufen. Der Zähler muss ebenfalls klein sein, daher wählen wir eine Binärzahl als Zähler.\\
		
		Phase 1:\\
		
		Überprüfe, ob das Eingabewort die Form $v \# w$ hat mit $v,w \in \{0,1\}^*$. Falls nicht, verwerfe. Falls es zutrifft, weiter mit Phase 2.\\
		Lauftzeit $O(n)$\\
		
		Phase 2:\\
		
		Setze die Position auf das erste Symbol von $v$. Sei $v=v_1...v_n$ und $w=w_1...w_m$. Setze eine zweite Spur ein, um den Zähler, der zwingend mitgeschoben wird, für die Position anzugeben, an der man gerade liest. Nun arbeite wie folgt:
		
		Es gibt mehrere Optionen für die Übergänge\\
		
		\begin{itemize}
			\item Liest man $v_i \in \{0,1\}$, vergleiche $v_i$ mit der $i$. Position rechts von $\#$. Falls ein B gelesen wird\\
			$\Rightarrow$ akzeptiere, ansonsten vergleiche das gelesene Symbol $w_i \in \{0,1\}$ mit $v_i$. Falls $v-i \neq w_i$ dann akzeptiere. 
			
			\item Liest man $v_i \in \{0,1\}$, darf der Lesekopf einen Schritt nach rechts bewegt werden. Erhöhe den Zähler um 1 und wähle erneut eine Option
			
			\item Liest man $\#$, gehe $n+1$ Schritte. Verwerfe, wenn ein $B$ gelesen wird (da dann $v=w$ gilt), ansonsten akzeptiere 
		\end{itemize}
		
		Die Optionen sind nicht deterministisch gewählt worden\\
		Laufzeit: $O(n*log(n))$\\
		Laufzeit insgesamt: $O(n*log(n))$ 
		
		\item (b) Nein so funktioniert das hier nicht, denn man muss jedes Symbol von $v$ und $w$ einzeln überprüfen und kann nicht bei Ungleichheit akzeptieren (der Schritt nach rechts darf so nicht ausgeführt werden).\\
	\end{enumerate}
	\section[a 8.2]{Aufgabe 8.2}
	\begin{enumerate}[label=(\alph*).]
		\item Graphen in Adjazenzschreibweise mit Colorierung:
		
		\begin{equation}
			\begin{array}{l | c | c | c | c | c | c | c}
				\ & \rt1 &  \rt2 & \bt3 & \bt4 & \rt5 & \bt6 & \rt7\\
				\hline
				\rt1& \rc0 & \rc0 & 1 & 0 & \rc0 & 0 & \rc0\\
				\rt2& \rc0 & \rc0 & 1 & 1 & \rc0 & 0 & \rc0\\
				\bt3& 1 & 1 & \bc0 & \bc0 & 1 & \bc0 & 0\\
				\bt4& 0 & 1 & \bc0 & \bc0 & 1 & \bc0 & 1\\
				\rt5& \rc0 & \rc0 & 1 & 1 & 0 & 1 & 0\\
				\bt6& 0 & 0 & \bc0 & \bc0 & 1 & 0 & 1\\
				\rt7& \rc0 & \rc0 & 0 & 1 & 0 & 1 & 0\\
			\end{array}
		\end{equation}
		Wie in der Matrix dargestellt können die Punkte erfolgreich 2-Coloriert werden.
		\begin{equation}
			\begin{array}{l | c | c | c | c | c | c | c}
				 & \rt1 &  \rt2 & \bt3 & \bt4 & \rt5 & \bt6 & \rt7\\
				\hline
				\rt1& \rc0 & \rc0 & 1 & 0 & \rc0 & 0 & \rc\textbf{\textcircled{1}}\\
				\rt2& \rc0 & \rc0 & 1 & 1 & \rc0 & 0 & \rc0\\
				\bt3& 1 & 1 & \bc0 & \bc0 & 1 & \bc0 & 0\\
				\bt4& 0 & 1 & \bc0 & \bc0 & 1 & \bc0 & 1\\
				\rt5& \rc0 & \rc0 & 1 & 1 & 0 & 1 & 0\\
				\bt6& 0 & 0 & \bc0 & \bc0 & 1 & 0 & 1\\
				\rt7& \rc\textbf{\textcircled{1}} & \rc0 & 0 & 1 & 0 & 1 & 0\\
			\end{array}
		\end{equation}
		Es kann keine korrekte 2-Färbung gefunden werden, da der Kreis um \{1,3,5,6,7\} eine ungerade Anzahl von Elementen hat. Somit lässt sich der gesamte graph nicht 2-Colorieren.
		\pagebreak
		\item \textbf{Algorithmus für 2-Colorability}:\\
		Nimm einen Knoten aus jeder Zusammenhangskomponente und speichere ihn in $Colored$\\
		\begin{lstlisting}
while Colored != V do
	for v in Colored do
		if (Ein Nachbarknoten besitzt die gleiche Farbe) then
			reject; 
		end
		Faerbe alle Nachbarknoten mit 
		der anderen Farbe (nicht c(v));
		Colored := Colored Union Nachbarknoten(v);  
	end	
end		
accept;		
		\end{lstlisting}
		
		Graphzusammenhangproblem lösbar in $O(n^2)$ für $n = \vert V \vert^2$. Die While-Schleife läuft maximal $\vert V\vert$ mal und die Forschleife ebenfalls maximal $\vert V\vert$ mal. Dabei werden maximal $\vert E\vert$ Kanten betrachtet. Insgesamt ergibt sich $O(\vert V\vert^2 \cdot \vert E\vert)$. Insgesamt ist der Algorithmus also durch $O(n^2 \cdot \vert E\vert)$ beschränkt.\\
		
		Korrektheit:\\
		Falls der Graph in 2 - Colorability liegt
		\begin{itemize}
			\item Der Algorithmus färbt in jeder Zusammenhangskomponente einen Knoten
			\item Iterativ werden deren Nachbarknoten immer weiter entgegengesetzt gefärbt
			\item Der Algorithmus akzeptiert den Graphen
		\end{itemize}	
		Falls der Graph nicht in 2 - Colorability liegt
		\begin{itemize}
			\item Der Algorithmus fäbt in jeder Zusammenhangskomponente einen Knoten
			\item Iterativ werden deren Nachbarknoten immer weiter entgegengesetzt gefärbt
			\item Es passiert, dass ein Knoten die gleiche Farbe wie sein Nachbar besitzt. 
			\item Der Algorithmus verwirft den Graphen
		\end{itemize}
	\end{enumerate}

	\section[a 8.3]{Aufgabe 8.3}
	Um zu zeigen, dass 3-Colorability in NP liegt wird 3-Colorability $\leq_p$ SAT gezeigt.\\
	Es gibt eine Funktion f mit folgenden Eigenschaften welche in polynomieller Zeit berechenbar ist.\\
	Jede Farbkonfiguration aus dem Problem der 3-Colorability wird folgendermaßen im SAT-Problem konfiguriert: 
	\begin{align*}
	\forall v \in V, x^1_v,x^2_v,x^3_v,\text{mit} \begin{cases}
	i = \text{farbe}(1...3)|  x^i_v = 1\\
	i \neq \text{farbe}(1...3)|  x^i_v = 0
	\end{cases}\\
	\text{Knotenbedingung: }& \bigwedge_{v \in V} (x^1_v \vee x^2_v \vee x^3_v)\\
	\text{Kantenbedingung: }& \bigwedge_{\{u,v\} \in E} (\bar{x}^1_u \vee \bar{x}^1_v) \wedge ( \bar{x}^2_v \vee \bar{x}^2_u) \wedge (\bar{x}^3_u \vee \bar{x}^3_v)\\
	\varphi = \text{Knotenbedingung $\wedge$ Kantenbedingung}\\
	f konstruiert \varphi\\
	Anzahl der Literale: O(3*|V|+3*|E|)\\
	max(|E|) = |V^2|\\
	Anzahl der Literale: O(|V^3|)\\
	\end{align*}
	Somit ist die Länge der zu berechnenden Formel für f polynomiell beschränkt und kann also in polynomieller Zeit berechnet werden.\\
	
	Korrektheit:\\
	\begin{align*}
		\text{Graph G ist colorierbar.} \Rightarrow& \text{Es gibt 3-Färbung.}\\
		\Rightarrow& \text{Knotenbedingung ist erfüllt, da jeder Knoten eine Farbe hat.}\\
		\Rightarrow& \text{keine zwei benachbarten Kanten haben die selbe Farbe.}\\
		\Rightarrow& \text{Kantenbedingung ist erfüllt.}\\		
		\Rightarrow& \varPhi \text{ ist erfüllbar}\\
	\end{align*}
	\begin{align*}
	\varphi \text{ ist erfüllbar.} \Rightarrow& \text{Es gibt eine erfüllende Belegung für }\varphi.\\
	\Rightarrow& \text{Die Knotenbedingung und die Kantenbedingung sind erfüllt}\\
	\Rightarrow& \text{jeder Knoten hat mindestens eine Farbe}\\
	\Rightarrow& \text{keine zwei adjazenten Knoten sind gleichfarbig}\\		
	\Rightarrow& \text{es gibt eine korrekte 3-Färbung für G}\\
	\end{align*}
	Somit lässt sich 3-Colorability auf SAT Reduzieren. Also ist auch 3-Colorability in NP.
	$\square$
\end{document}