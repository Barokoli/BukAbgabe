\documentclass{article}

\usepackage[utf8]{inputenc}
\usepackage[T1]{fontenc}
\usepackage{lmodern}
\usepackage{tikz}
\usepackage{listings}
\usepackage{color}
\usepackage{enumitem}
\usepackage{mathtools}
\usepackage{amsmath}
\usepackage{amsfonts}
\newlist{arrowlist}{itemize}{1}
\setlist[arrowlist]{label=$\Rightarrow$}

\title{BuK Abgabe 9 | Gruppe 17}
\author{Malte Meng (354529) , Charel Ernster (318949), Sebastian Witt (354738)}
\begin{document}
	\maketitle 
	\section[a 9.1]{Aufgabe 9.1}
	\begin{enumerate}[label=(\alph*).]
		\item \textbf{MaxSpanTree}\\
		Sei \{V,E\} der Graph G und $E^G$ die Menge der Gewichte von G. n = |E|\\\\
		\textbf{Zertifikat:}\\
		Z ist das Zertifikat der Form: $E_1\#E_2\#E_3\#...E_j$\\
		$E_i$ Index einer Kante des Spannbaumes mit i $\in$ [1...j], j $\leq$ n\\\\
		\textbf{Vertifizierer:}\\
		Der Vertifizierer verifiziert das Zertifikat in polynomieller Zeit folgendermaßen ($l$ = länge Zertifikat $l \leq 2n$):\\
		\begin{itemize}
			\item Prüfe ob Zertifikat in der Form "$E_1\#E_2\#E_3\#...\#E_j$" ist. \begin{flushright}O($2n$)\end{flushright}
			\item Prüfe ob $\{E_1,...,E_n\}$ ein Spannbaum von G ist. Prüfe ob alle v aus V verbunden sind. 
			\begin{flushright}O($n$)\end{flushright}
			\item Prüfe ob $\sum_{i=0}^{j}E^G_i \geq c$ 
			\begin{flushright}O($n$)\end{flushright}
		\end{itemize}
		Somit kann das Zertifikat in polynomieller Zeit O(4n) verifiziert werden.\\
		Es verifiziert, dass es einen Spannbaum mit Gewicht $\geq$ c in G gibt.
		Das Entscheidungsproblem ist somit in NP.\begin{flushright}$\square$\end{flushright}
		
		\item \textbf{Composite}\\
		\textbf{Zertifikat:}\\
		$k$ ist keine Primzahl, somit existiert eine Zerlegung T mit $\prod_0^n T_i = k , n \in \mathbb{N}, T_i \in Primzahlen$ \\
		Z ist das Zertifikat der Form: $T_1\#T_2\#...\#T_n$.\\\\
		\textbf{Verifizierer:}
		Der Vertifizierer verifiziert das Zertifikat in polynomieller Zeit folgendermaßen ($l$ = länge Zertifikat $l \leq 2n$):\\
		\begin{itemize}
			\item Prüfe ob Zertifikat in der Form "$T_1\#T_2\#T_3\#...\#T_n$" ist. \begin{flushright}O($2n$)\end{flushright}
			\item Prüfe ob $\prod_{i=0}^{n} T_i = k$
			\begin{flushright}(polynomiell lösbar)\end{flushright}
		\end{itemize}
		Somit kann Z in polynomieller Zeit verifiziert werden. Es verifiziert ob das Zertifikat eine Primzahlzerlegung von der binärkodierten Zahl k ist.
		\begin{flushright}$\square$\end{flushright}


		\item \textbf{Graphisomorphie}\\
		\textbf{Zertifikat:}\\
		Sei $\delta$ die Permutation der Indizes der Menge $V_1$ die so der Abbildung f $V_1 \rightarrow V_2$ entspricht, so dass $(v_i,v_j) \in E_1 \implies (f(v_i),f(v_j)) \in E_2$\\
		das Zertifikat Z ist nun diese Permutation in Tupelschreibweise. \\
		$\delta = (p_1,p_2,...,p_n), Z = p_1\#p_2\#...\#p_n$\\\\
		\textbf{Verifizierer:}\\
		Der Verifizierer Zertifiziert das Zertifikat in polynomieller Zeit folgendermaßen:
		\begin{itemize}
			\item Prüfe ob Zertifikat in der Form "$p_1\#p_2\#p_3\#...\#p_n$" ist. \begin{flushright}O($2n$)\end{flushright}
			\item Prüfe ob $$
			\forall (v_i,v_j) \in E_1\ \exists i' = p_i, j' = p_j, (v_{i'},v_{j'}) \in E_2
			$$
			\begin{flushright}O(n)\end{flushright}
		\end{itemize}
		Somit kann in polynomieller Zeit verifiziert werden ob das Zertifikat stimmt, ob also eine Permutation existiert die die Knoten von $G_1$ auf $G_2$ abbildet und Kanten erhalten bleiben. Also prüft es ob die Graphen isomorph sind. $\Rightarrow$ Graphisomorphie $\in$ NP.
		\begin{flushright}$\square$\end{flushright}
	\end{enumerate}

	\section[a 9.3]{Aufgabe 9.3}
	\textbf{DoubleSAT}:\\
	\begin{enumerate}[label=\Roman*.]
		\item DoubleSAT ist in NP da ein Verifizierer leicht ein Zertifikat verifizieren kann welches der Form zwei gültiger Belegungen sind. Somit $DoubleSAT \in NP$
		\item Reduktion von SAT auf DoubleSAT:\\
		Es gibt eine polynomiell berechenbare Funktion f, die SAT auf DoubleSAT abbildet.\\
		Diese Funktion f bildet aus den Literalen von SAT neue Literale für DoubleSAT folgendermaßen:\\
		KNF von SAT: $((l_1) \wedge (l_2) \wedge ... (l_n))\\   \rightarrow $\\
		DoubleSAT: $ ((l_1 \vee x) \wedge (l_2 \vee x) \wedge ... (l_n \vee x) \wedge 
		(l_1 \vee \bar{x}) \wedge (l_2 \vee \bar{x}) \wedge ... (l_n \vee \bar{x}))$\\
		\item Korrektheit:\\
		AC entspricht der KNF der Form $((l_1 \vee x) \wedge (l_2 \vee x) \wedge ... (l_n \vee x)$\\
		$A\neg C$ entspricht der KNF der Form $(l_1 \vee \bar{x}) \wedge (l_2 \vee \bar{x}) \wedge ... (l_n \vee \bar{x}))$\\
		A entspricht der KNF der Form $((l_1) \wedge (l_2) \wedge ... (l_n))$
		\\\\
		Es gibt zwei Belegungen für $\gamma \Rightarrow$ es gibt eine Belegung für $\delta$\\
		\begin{equation}
		\nonumber
			\begin{split}
				\text{Es gibt zwei Belegungen für }\gamma \Rightarrow & AC \wedge A \neg C \text{ hat zwei Belegungen}\\
				\Rightarrow & A \text{ hat eine Belegung}\\
				\Rightarrow & \text{ es gibt eine Belegung für }\delta
			\end{split}
		\end{equation}
		
		Es gibt eine Belegung für $\delta \Rightarrow$ es gibt zwei Belegungen für $\gamma$\\
		\begin{equation}
		\nonumber
		\begin{split}
		\text{Es gibt eine Belegung für }\delta \Rightarrow & A \text{ hat eine Belegung}\\
		\Rightarrow & AC \wedge A\neg C \text{ hat zwei Belegungen}\\
		\Rightarrow & \text{ es gibt zwei Belegungen für }\gamma
		\end{split}
		\end{equation}
		\begin{flushright}$\square$\end{flushright}
	\end{enumerate}
	Somit gilt SAT $\leq_q$ DoubleSAT und DoubleSAT ist NP-vollständig.
	
\end{document}
