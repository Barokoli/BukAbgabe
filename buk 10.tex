\documentclass{article}

\usepackage[utf8]{inputenc}
\usepackage[T1]{fontenc}
\usepackage{lmodern}
\usepackage{tikz}
\usepackage{listings}
\usepackage{color}
\usepackage{enumitem}
\usepackage{mathtools}
\usepackage{amsmath}
\usepackage{amsfonts}
\newlist{arrowlist}{itemize}{1}
\setlist[arrowlist]{label=$\Rightarrow$}

\title{BuK Abgabe 9 | Gruppe 17}
\author{Malte Meng (354529) , Charel Ernster (318949), Sebastian Witt (354738)}
\begin{document}
	\maketitle 
	\section[a 10.1]{Aufgabe 10.1}
	\textbf{ Entscheidungsvariante:}\\
	
	Sei $b \in \mathbb{N}$ fest.\\
	
	$\exists j:\{1,...,n\}\to\{1,...,.m\}: \max\limits_{1 \leq i \leq m} \sum\limits_{j:s(j)=i} p_j=b$\\
	
\textbf{Reduktion:}	\\

Wir definieren:\\

$f: \sum^* \to \sum^*, x \mapsto \left\{\begin{array}{cl}

(2,(a_1,...,a_n),p-b)& ,x=(a_1,...,a_n),b,b\leq p-b \\ (2,(a_1,...,a_n),b) & , x=(a_1,...,a_n),b,b >p-b \\ x & ,\mbox{sonst} \end{array} \right.$

wobei $p=\sum_{i=1}^n a_i$. $f$ ist polynomiell berechenbar, denn die Berechnung von $p$ geht nach logarithmischen Kostenmaß in $\mathcal{O}(n*log(l))$, wobei $l=\max \{m \in \mathbb{N} | m= \vert bin(a_i) \vert , 1\leq i \leq n\}$.\\
Bleibt zu zeigen:\\

$x \in$ SubsetSum $ \Leftrightarrow f(x) \in$ MSE\\


wobei\\
MSE$:=\{(m,(p_1,...,p_m),b) | \exists j : \{1,...,n\} \to \{1,...,m\}: \max\limits_{1 \leq i \leq m} \sum\limits_{j:s(j)=i} p_j=b \}$\\
die von der obigen definierten Entscheidungsvariante erzeugte Sprache ist.\\

\textbf{Korrektheit}:\\

Nach Definition von SubsetSum gilt:\\
$x \in $SubsetSum$ \Leftrightarrow x=(a_1,...,a_n),b$ mit $a_i,b \in \mathbb{N}, 1\leq i \leq n$ : $\exists I \subseteq \{1,...,n\}:\sum\limits_{i \in I} a_i=b$.\\

$\Leftrightarrow$ Definiere $J:=\{1,...,n\} \setminus I, s:\{1,...,n\} \to \{1,2\}, x \mapsto \left\{\begin{array}{cl} 1 & ,x\in I\\ 2 & , x\in J \end{array} \right.$\\

wohldefiniert, da $I,J$ Partion von $\{1,...,n\}$ ist. Es gilt:\\

$\Leftrightarrow\max\limits_{1 \leq i \leq m} \sum\limits_{j:s(j)=i} a_j = \left\{\begin{array}{cl} 
(2,(a_1,...,a_n),p-b)& ,x=(a_1,...,a_n),b,b\leq p-b \\ (2,(a_1,...,a_n),b) & , x=(a_1,...,a_n),b,b >p-b\end{array} \right. \in $MSE\\

$\Leftrightarrow f(x) \in$ MSE
	
\section{Aufgabe 10.2}

3-Partition ist $\in$ NP, da Spezialfall von SubsetSum ist. 3-Partition ist NP-hart:\\
Zeige SubsetSum $\leq p$ 3-Partition, Eingabe SubsetSum: $a_1,...,a_n \in \mathbb{N}, b \in \mathbb{N}$.\\
Bilde SubsetSum Eingabe auf 3-Partition ab:\\

$a_1\prime,...,a_{n+3}\prime$ wobei $a_i\prime=a_i$ für $1<i<n$\\

$a_{n+1}\prime = 2A-b$, $a_{n+2}\prime = A+b$, $a_{n+3}\prime = 2A$\\

In der Summer ergibt die Eingabe $6A$. Suche eine Aufteilung in drei Teilmengen, die jeweils in der Summer $2A$ ergeben. Die Eingabetransformation ist polynomiell berechenbar.\\

\textbf{Korrektheit.}\\

$\exists$ Lösung $P$ für $3-P$\\
$\Rightarrow a_{n+1}\prime, a_{n+2}\prime$ und $a_{n+3}\prime$ müssen in verschiedene Teilmengen sein, da $a_{n+1}\prime+a_{n+2}\prime>2A$, $a_{n+1}\prime+a_{n+3}\prime>2A$ und $a_{n+2}\prime+a_{n+3}\prime>2A$\\
$\Rightarrow \exists$ Lösung $S$ für SubsetSum, da $a_{n+1}\prime+\sum\limits_{i \in P} a_i = b$\\

$\exists$ Lösung $S$ für SubsetSum\\
$\Rightarrow S\cap \{a_{n+1}\prime\}=S\prime \subset \{a_{n+1}\prime,...,a_{n+3}\prime \}$ mit $\sum\limits_{i \in S\prime} a_i=2A$\\
$\Rightarrow$ $S\prime$ ist die Lösung für 3-Partition
	
	
\end{document}
