\documentclass{article}

\usepackage[utf8]{inputenc}
\usepackage[T1]{fontenc}
\usepackage{lmodern}
\usepackage{tikz}
\usepackage{listings}
\usepackage{color}
\usepackage{enumitem}
\usepackage{mathtools}
\usepackage{amsmath}
\usepackage{amsfonts}

\title{BuK Abgabe 5 | Gruppe 17}
\author{Malte Meng (354529) , Charel Ernster (318949), Sebastian Witt (354738)}
\begin{document}
	\maketitle 
	\section[a 5.1]{Aufgabe 5.1}
		\begin{enumerate} [label=(\alph*).]
			\item Gegeben ist:\\
			$L_1 \leq L_2$ und $L_2 \leq L_3$
			$\rightarrow$
			$\exists f_1,f_2$ mit:\\
			$f_{1|2}$ bildet alle Ja/Nein-Instanzen von $L_{1|2}$ auf Ja/Nein-Instanzen von $L_{2|3}$.\\
			Somit gibt es die Bildmenge $M_1$ der Ja/Nein-Instanzen in $L_2$ von der Abbildung $f_1$ ($L_1 \xrightarrow{f_1} L_2$).\\
			$M_1 \subseteq L_2 \Rightarrow\\ \exists M_2$ mit $M_2 = f_2(M_1)$ und $M_2 \subseteq L_3 \Rightarrow\\
			\exists f_3$ mit $f_3 = L_1 \xrightarrow{L_1 \rightarrow M_1 \rightarrow M_2} M_2$ mit $ M_2 \subseteq L_3 \Rightarrow\\
			f_3 := L_1 \rightarrow L_3$ mit $f_3$ bildet alle Ja/Nein-Instanzen von $L_1$ auf $L_3$ ab. Die Korrektheit der Funktionen bleibt wie die Ursprünglichen $f_1,f_2$. Somit gilt $L_1 \leq L_3$ für beliebige $L_1,L_2,L_3$ mit $L_1 \leq L_2$ und $L_2 \leq L_3$. Das Reduktionskonzept ist also transitiv.  $\square$
			\item $L_1 \leq L_2$ $\Rightarrow$ $\exists f = \begin{cases}
				L_{ja} \rightarrow M_{ja}\\
				L_{nein} \rightarrow M_{nein}\\
			\end{cases}\\$
			$L_{ja} = \overline{L}_{nein}$ , $L_{nein} = \overline{L}_{ja}$ , $M_{ja} = \overline{M}_{nein}$ , $M_{nein} = \overline{M}_{ja} \Rightarrow$\\
			$\exists \overline{f} = \begin{cases}
				\overline{L}_{nein} \rightarrow \overline{M}_{nein}\\
				\overline{L}_{ja} \rightarrow \overline{M}_{ja}\\
			\end{cases}\Rightarrow \overline{L_1} \leq \overline{L_2}\\$ Somit gilt $L_1 \leq L_2 \Rightarrow \overline{L_1} \leq \overline{L_2}$  $\square$
		\end{enumerate}
\end{document}