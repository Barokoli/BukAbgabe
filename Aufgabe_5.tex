\documentclass{article}

\usepackage[utf8]{inputenc}
\usepackage[T1]{fontenc}
\usepackage{lmodern}
\usepackage{tikz}
\usepackage{listings}
\usepackage{color}
\usepackage{enumitem}
\usepackage{mathtools}
\usepackage{amsmath}
\usepackage{amsfonts}

\title{BuK Abgabe 5 | Gruppe 17}
\author{Malte Meng (354529) , Charel Ernster (318949), Sebastian Witt (354738)}
\begin{document}
	\maketitle 
	\section[a 5.1]{Aufgabe 5.1}
		\begin{enumerate} [label=(\alph*).]
			\item Gegeben ist:\\
			$L_1 \leq L_2$ und $L_2 \leq L_3$
			$\rightarrow$
			$\exists f_1,f_2$ mit:\\
			$f_{1|2}$ bildet alle Ja/Nein-Instanzen von $L_{1|2}$ auf Ja/Nein-Instanzen von $L_{2|3}$.\\
			Somit gibt es die Bildmenge $M_1$ der Ja/Nein-Instanzen in $L_2$ von der Abbildung $f_1$ ($L_1 \xrightarrow{f_1} L_2$).\\
			$M_1 \subseteq L_2 \Rightarrow\\ \exists M_2$ mit $M_2 = f_2(M_1)$ und $M_2 \subseteq L_3 \Rightarrow\\
			\exists f_3$ mit $f_3 = L_1 \xrightarrow{L_1 \rightarrow M_1 \rightarrow M_2} M_2$ mit $ M_2 \subseteq L_3 \Rightarrow\\
			f_3 := L_1 \rightarrow L_3$ mit $f_3$ bildet alle Ja/Nein-Instanzen von $L_1$ auf $L_3$ ab. Die Korrektheit der Funktionen bleibt wie die Ursprünglichen $f_1,f_2$. Somit gilt $L_1 \leq L_3$ für beliebige $L_1,L_2,L_3$ mit $L_1 \leq L_2$ und $L_2 \leq L_3$. Das Reduktionskonzept ist also transitiv.  $\square$
			\item $L_1 \leq L_2$ $\Rightarrow$ $\exists f = \begin{cases}
				L_{ja} \rightarrow M_{ja}\\
				L_{nein} \rightarrow M_{nein}\\
			\end{cases}\\$
			$L_{ja} = \overline{L}_{nein}$ , $L_{nein} = \overline{L}_{ja}$ , $M_{ja} = \overline{M}_{nein}$ , $M_{nein} = \overline{M}_{ja} \Rightarrow$\\
			$\exists \overline{f} = \begin{cases}
				\overline{L}_{nein} \rightarrow \overline{M}_{nein}\\
				\overline{L}_{ja} \rightarrow \overline{M}_{ja}\\
			\end{cases}\Rightarrow \overline{L_1} \leq \overline{L_2}\\$ Somit gilt $L_1 \leq L_2 \Rightarrow \overline{L_1} \leq \overline{L_2}$  $\square$
		\end{enumerate}
	\section[a 5.3]{Aufgabe 5.3}
		\begin{enumerate} [label=(\alph*).]
			\item Die Sprache $A_{62}$ ist rekursiv aufzählbar aber nicht rekursiv.\\
			Beweis in zwei Schritten durch reduktion von $H_\epsilon$ auf $A_{62}$:\\
			\begin{enumerate}[label=\Roman*.]
				\item $A_{62}$ ist nicht rekursiv:
				\begin{enumerate}[label=]
					\item Es gibt eine berechenbare Funktion f mit folgenden Eigenschaften:\\
					\begin{enumerate}[label=$\bullet$]
						\item $\langle M \rangle$ ist keine gültige Gödelnummer f($\langle M \rangle$) = $\overline{w}$  mit  $\overline{w} \in \overline{H_{\epsilon}}$
						\item Falls w = $\langle M \rangle$ für eine TM M, so sei f(w) die Gödelnummer einer TM $M_{neu}$ mit folgenden Eigenschaften:\\
						$M_{neu}$ prüft Eingabelänge l
							$\begin{cases}
							l > 62$ | verwerfe$\\
							l \leq 62 $ | verwerfe die Eingabe und Simuliere M mit Eingabe $\epsilon\\
							\end{cases}$
					\end{enumerate}
					\item f bildet $H_{\epsilon}$ korrekt auf $A_{62}$ ab:\\
					Falls w keine Gödelnummer ist, so ist die Korrektheit klar, denn in diesem Fall gilt W $\notin H_{\epsilon}$ und f(w) $\notin A_{62}$\\
					Sei w = $\langle M \rangle$ für eine TM M und sei f(w) = $\langle M_{neu} \rangle$\\
					Es gilt:\\
					$w \in H_{\epsilon} \Rightarrow$ M hält auf $\epsilon$\\
					$\Rightarrow \langle M_{neu} \rangle$ hält auf jeder Eingabe mit l $\leq$ 62\\
					$\Rightarrow \langle M_{neu} \rangle$ hält auf mindestens zwei Wörtern der Länge höchstens 62\\
					$\Rightarrow \langle M_{neu} \rangle \in A_{62}$\\
					$\Rightarrow f(w) \in A_{62}$
					\\\\	
					$w \notin H_{\epsilon} \Rightarrow$ M hält nicht auf $\epsilon$\\
					$\Rightarrow \langle M_{neu} \rangle$ hält auf keiner Eingabe\\
					$\Rightarrow \langle M_{neu} \rangle \notin A_{62}$\\
					$\Rightarrow f(w) \notin A_{62}$\\
					Daher ist $H_{\epsilon} \leq A_{62} \Rightarrow A_{62}$ ist nicht Rekursiv.  
				\end{enumerate}
			\item $A_{62}$ ist aufzählbar:\\
			Es gibt eine TM M die $A_{62}$ erkennt mit folgenden Eigenschaften:
			\begin{enumerate}[label=$\bullet$]
				\item M Simuliert alle Gödelnummern parallel. Also jeweils einen Schritt auf allen Nummern pro Iteration.
				\item Die Simulation funktioniert indem jeweils wieder ein Schritt auf jedem Wort mit l $\leq$ 62 Simuliert wird.
				\item Werden dabei zwei Wörter Akzeptiert "druckt" M die korrespondierende Gödelnummer.
			\end{enumerate}
			\end{enumerate}
			\item Beweis dass $B_1$ nicht rekursiv aufzählbar ist:
			\begin{equation*}
			\begin{aligned}
			\overline{B_1} = \{\langle M \rangle\textnormal{ | M akzeptiert kein Wort}\} &\geq
			\{\langle M \rangle\textnormal{ | M hält auf keinem Wort}\} = \overline{H_{all}}\\
			\overline{B_1} &\geq \overline{H_{all}}\\
			H_{all} \textnormal{ist nicht rekursiv aufzählbar } &\Rightarrow \overline{H_{all}}\textnormal{ist nicht rekursiv aufzählbar }\\
			&\Rightarrow \overline{B_1}\textnormal{ ist nicht rekursiv aufzählbar}\\
			&\Rightarrow \overline{B_1} \textnormal{ ist nicht rekursiv aufzählbar}\\
			&\Rightarrow B_1 \textnormal{ ist nicht rekursiv aufzählbar } \square
			\end{aligned}
			\end{equation*}
		\end{enumerate}
\end{document}