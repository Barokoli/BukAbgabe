\documentclass{article}

\usepackage[utf8]{inputenc}
\usepackage[T1]{fontenc}
\usepackage{lmodern}
\usepackage{tikz}
\usepackage{listings}
\usepackage{color}
\usepackage{enumitem}
\usepackage{mathtools}
\usepackage{amsmath}
\usepackage{amsfonts}

\title{BuK Abgabe 3 | Gruppe 17}
\author{Malte Meng (354529) , Charel Ernster (318949), Sebastian Witt (354738)}
\begin{document}
	\maketitle 
	\section[a 3.1]{Aufgabe 3.1}
	Die Turingmaschine funktioniert nach folgendem Muster:\\
	Maschine besitzt vier Zustände um die Anzahlen von 0 und 1 zu zählen:\\ $(q0odd,q1odd,q0even,q1even)$ , es gibt noch eine Schrittgröße die bei 1 anfängt und pro Iteration verdoppelt wird, außerdem wird noch die Schrittrichtung gespeichert und wechselt nach jeder Iteration.
	Jede Iteration läuft folgendermaßen ab:
	\begin{enumerate}
		\item Gehe in Schrittrichtung mit der Schrittgröße, wenn dass Zeichen währenddessen unter dem Kopf wechselt merke dir ob Zustand odd oder even ist. Nach jedem erfolgreichen Schritt wird der Zustand gewechselt.
		\item Gehe in Schrittrichtung mit der Schrittgröße weiter bis ein Blank erreicht wird. Ist dann der Zustand nicht gleich dem gemerkten Zustand, wird das Wort abgelehnt. Nach jedem erfolgreichen Schritt wird der Zustand gewechselt.
		\item Ändere die Schrittrichtung, verdopple die Schrittlänge und setzte die Zustände auf Odd. Nächste Iteration.
	\end{enumerate}
	 Der Algorithmus beendet erfolgreich, wenn die Schrittlänge länger ist als die Anzahl von "0" bzw "1", also während des ersten Schrittes das Vorzeichen gewechselt wird.  (Ausnahme bei $L = 0^* | L = 1^*$ vor Wechsel wird bereits blank erreicht)\\
	 Das Grundkonzept funktioniert indem verglichen wird ob:\\ $(n_0 mod 2^i > 0) == (n_1 mod 2^i > 0)$\\ nach jeder Iteration i\\\\
	 Der Algorithmus hat den Zeitbedarf $O(m$ log $m)$, da pro Iteration m Schritte gemacht werden und der Algorithmus im worst-case nach log $m$ Schritten terminiert, wenn Schrittlänge > $n_0$.\pagebreak
	 \section[a 3.2]{Aufgabe 3.2}
	 \lstset{
	 	numbers=left,
	 	stepnumber=1,     
	 	numberfirstline=false
	 }
	 \begin{lstlisting}[title="RAM Zweierlogarithmus"]
		 CLOAD 2
		 STORE 2
		 CLOAD n
		 STORE 1
		 IF c(0) = 0 THEN GOTO 13
		 LOAD 1
		 DIV 2
		 STORE 1
		 LOAD 3
		 CADD 1
		 STORE 3
		 GOTO 5
		 LOAD 3
		 STORE 1
		 END
	 \end{lstlisting}
	 Die RAM Läd zunächst die Eingabe an erste Stelle.
	 Dann wird in jedem Schritt die Eingabe halbiert, bis sie Leer ist.
	 Die Schritte werden in Register 3 gespeichert. 
	 Somit wird gezählt an welcher Stelle das höchste bit ist, welches die Zweierpotenz bestimmt.
	 Ist die Stelle bestimmt, wird der Zähler ins erste Register geladen und ausgegeben.
	 \section[a 3.3]{Aufgabe 3.3}
	 Idee: Äquivalenz von $\mathbb{N} = \mathbb{N}^*$ zeigen.\\\\
	 $a.b = a*10^n+b$ mit $a,b \in \mathbb{N} \land n = $ Anzahl Stellen von b.\\
	 $\implies \mathbb{N}.\mathbb{N} \in \mathbb{N}$ "." steht für die Konkatenation bzw 1.2 = 12.\\
	 $\implies \mathbb{N}.\mathbb{N}.\mathbb{N}... \in \mathbb{N} \implies \mathbb{N} = \mathbb{N}^*$ \\
	 $\mathbb{N}$ ist abzählbar somit auch $\mathbb{N}^*$
	\section[3.4]{Aufgabe 3.4}
	\begin{enumerate}
		\item Konstruiere eine TM die $\langle M \rangle w$ simuliert. 
		Simuliere 42 Schritte und akzeptiere, wenn bis dahin $\langle M \rangle w$ hält.$\square$\\
		\item Annahme: Es gibt eine Maschine die $H_{\leq42}$ entscheidet.\\
		Somit lässt sich eine Maschine $M_H$ mit Unterprogramm $H_{\geq42}$ konstruieren, so dass $M_H$ das Halteproblem entscheidet.\\
		$M_H$ ist wie folgt aufgebaut:\\
		Simuliere die ersten 42 Schritte mit $H_{\leq42}$ aus a). Wenn in die Maschine in den ersten 42 Schritten stoppt akzeptiere. Wenn nicht wird das Wort an $H_{\geq42}$ weitergeleitet. Akzeptiert/Ablehnt diese Akzeptiert/Ablehnt auch $M_H$.\\
		Diese Maschine entspricht dem Halteproblem:\\\\
		$M_H = \begin{cases}
			Schritte \leq 42$ | Akzeptiere wenn hält, sonst nicht$\\
			Schritte > 42$ | Akzeptiere wenn hält, sonst nicht$
			\end{cases} \Bigg \} = H$\\\\
		Da das Halteproblem, wie in der Vorlesung gezeigt, nicht entscheidbar ist, kommt es zum Widerspruch. Somit kann es keine Maschine $H_{\geq42}$ geben. $\square$
	\end{enumerate}	 
\end{document}